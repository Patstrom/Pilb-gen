The given differential equation that models the curvature is as follow:

\begin{center}
    %Curvature of the rulerbow.
$\frac{d^2y}{dx^2} + qy * [1+ (\frac{dy}{dx})^2]^{\frac{3}{2}} = 0$
\end{center}

\subsection{Initial values}
The ruler is one meter long and at is drawn such that the edges are 0.3m away from the center along the y-axis. Thus our initial values are $y(0)=0.3$, $y'(0)=0$ and $y(a)=0$. Since the bow is symmetrical around the y-axis we can also conclude that the arc length between $x=0$ and $x=a$ is $0.5$. We also only need to consider this interval. 

\subsection{Approach}
The general algorithm is to take first determine the relationship $a(q)$ using the simplified equation. Using this relationship we can also decide a reasonable interval for $q$. We then use the lower and upper bound for $a$ as our starting guesses for $q$ and calculate $a$ for every $q$. We then use these values to calculate the arc length and if it is satisfyingly close to 0.5 we have found our values for $q$ and $a$. If it is not we guess a new value for $q$.

\subsection{Simplified equation}
When we consider the equation at $x=0$ we can omit the $y'$ term. We are given a possible solution $y(x)=0.3*cos(\sqrt{q}*x)$. To prove this is correct we do the following:

\begin{center}
    $y(x)=0.3cos(\sqrt{q}*x)$ \\ 
$y'(x)=-0.3sin(\sqrt{q}*x)*\sqrt(q)$ \\ 
$y''(x)=-0.3cos(\sqrt{q}*x)*q$ \\
$y'' + qy = 0 \Rightarrow -0.3cos(\sqrt{q}*x)*q+q*0.3cos(\sqrt{q}*x) = 0$ \\
QED    
\end{center}

This in turn gives us the relationship $a(q)$:

\begin{center}
    $y(x)=0.3*cos(\sqrt{q}*x)$ \\
$y(a) = 0 \Rightarrow y(a)=0.3*cos(\sqrt{q}*a) = 0$ \\
$\Rightarrow cos(\sqrt{q}*a) = 0$ \\
$\Rightarrow \sqrt{q}*a = \pi * n - \frac{\pi}{2}      n \in Z $ \\
We only consider $n=1$ \\
$\sqrt{q}*a=\frac{\pi}{2} \Rightarrow a=\frac{\pi}{2*\sqrt{q}}$
\end{center}

It follows that:

\begin{center}
    \begin{equation}
  q=(\frac{pi}{2*a})^2
\end{equation}

\end{center}

\subsection{Interval for q}
Using the pythagorean theorem we can conclude that the upper bound for $a$ is $\sqrt{0.3^2+x^2}=0.5 \Rightarrow x=0.4$. The concave function representing the bow will have a longer arc length than the linear function intersecting the x-axis at the same x-value for any given $x$ within our constraints. An upper bound of $0.4$ for $a$ means the lower value for $q$ is $15.4$. 

For the lower bound of $a$ we simply use 0. Since the lower bound for $a$ is used to calculate the upper bound of $q$ this does not matter a whole lot. We expect to find our value for $a$ closer to the upper bound rather than the lower.

\subsection{arc length}
Starting at $q=0$ we calculate $a(q)$ and $arcLength(q, a(q))$. If $abs(arcLength(q, a(q))-0.5) < 1.e-5$ we terminate our algorithm since we have found the sought precision of the arc length. If the condition is not met we increment $q$ by 0.001 and start oer.

\subsection{Force acting upon the bow}
Since the lever arm is 0.3 we get a trivial equation:

$0.3*S=-q*y(0) \Rightarrow 0.3*S=-q*0.3 \Rightarrow S=-q$