The given differential equation that models the curvature is as follow:

\begin{center}
    %Curvature of the rulerbow.
\begin{equation}
  \frac{d^2y}{dx^2} + qy * [1+ (\frac{dy}{dx})^2]^{\frac{3}{2}} = 0
\end{equation}

\end{center}

\subsection{Initial values}
The ruler is one meter long and at is drawn such that the edges are 0.3m away from the center along the y-axis. Thus our initial values are $y(0)=0.3$, $y'(0)=0$ and $y(a)=0$. Since the bow is symmetrical around the y-axis we can also conclude that the arc length between $x=0$ and $x=a$ is $0.5$. We also only need to consider this interval. 

\subsection{Approach}
The general algorithm is to take a guess for the value of $q$, solve the differential equation with this value of $q$ to get the x-value right before the curve intersects the x-axis and the x-value right after. With these two guesses for a potential $a$ we interpolate a more precise value of $x$ at the intersection. If this value yields an arc length satisfyingly close to 0.5 we terminate our algorithm and plot the curve with our new-found value.

Our approach to solving this problem was to first of all solve the simplified equation at $x=0$ to find out the relationship between our two unknowns, q and a. We then used this relationship to determine a reasonable interval of guesses for $q$.

For every guess of $q$ we use a built in Matlab function that numerically solves a nonstiff differential equation. We then get our two guesses for $a$; One slightly smaller than the actual value and one slightly larger. Using the lower of these values we calculate the arc length. If $|arcLength-0.5| < 1e-5$ we have found a value of $a$ we consider precise enough. If the condition is not met we interpolate a more precis value for $a$ using the secant method. We do this for a total of 5 steps and for every step we re-calculate the arc length and check if it is precise enough. If we haven't found a precise enough value in 5 steps we choose a new value for $q$ and start over.

\subsection{Simplified equation}
When we consider the equation at $x=0$ we can omit the $y'$ term. We are given a possible solution $y(x)=0.3*cos(\sqrt{q}*x)$. To prove this is correct we do the following:

\begin{center}
    $y(x)=0.3cos(\sqrt{q}*x)$ \\ 
$y'(x)=-0.3sin(\sqrt{q}*x)*\sqrt(q)$ \\ 
$y''(x)=-0.3cos(\sqrt{q}*x)*q$ \\
$y'' + qy = 0 \Rightarrow -0.3cos(\sqrt{q}*x)*q+q*0.3cos(\sqrt{q}*x) = 0$ \\
QED    
\end{center}

This in turn gives us the relationship $a(q)$:

\begin{center}
    \begin{equation}
  y(x)=0.3*cos(\sqrt{q}*x)
\end{equation}
\begin{equation}
  y(a) = 0 \Rightarrow y(a)=0.3*cos(\sqrt{q}*a) = 0
\end{equation}
\begin{equation}
  \Rightarrow cos(\sqrt{q}*a) = 0
\end{equation}
\begin{equation}
  \Rightarrow \sqrt{q}*a = \pi * n - \frac{\pi}{2}      n \in Z
\end{equation}
We only consider $n=1$
\begin{equation}
  \sqrt{q}*a=\frac{\pi}{2} \Rightarrow a=\frac{\pi}{2*\sqrt{q}}
\end{equation}

\end{center}

It follows that:

\begin{center}
    \begin{equation}
  q=(\frac{pi}{2*a})^2
\end{equation}

\end{center}

\subsection{Interval for q}
Using the pythagorean theorem we can conclude that the upper bound for $a$ is $\sqrt{0.3^2+x^2}=0.5 \Rightarrow x=0.4$. The concave function representing the bow will have a longer arc length than the linear function intersecting the x-axis at the same x-value for any given $x$ within our constraints. An upper bound of $0.4$ for $a$ means the lower value for $q$ is $15.4$. 

To find the lower bound of $a$ TODO: DET HÄR ANTAR JAG.

\subsection{Finding the intitial guesses for a}
TODO: Ode45 förklara F2


\subsection{Interpolating and arc length}
Using the lower of the initial guesses for a we calculate the arc length and check if $|arcLength-0.5| < 1e-5$. If the condition is not met we complete one step of the secant method to interpolate a more precise value and redo the calculation for arc length and check the condition again. We continue this process for a total of 5 steps. If the condition is not met in any of these steps we increment $q$ by 0.001 and start over.

\subsection{Force acting upon the bow}
Since the lever arm is 0.3 we get a trivial equation:

$0.3*S=-q*y(0) \Rightarrow 0.3*S=-q*0.3 \Rightarrow S=-q$